\documentclass[a4paper]{article}
\usepackage{graphicx}
\usepackage[english,russian]{babel}
\usepackage[utf8]{inputenc}
\usepackage[T2A]{fontenc}
\usepackage{tabto}
\usepackage{amsmath}
\usepackage{pgfplots}
\usepackage[left=25mm, top=20mm, right=10mm, bottom=10mm, nohead, nofoot]{geometry}
\begin{document}
    \begin{center} 
        \LARGE МАтан\\
        \LARGE диффуры\\
    \end{center}
    \newpage
    №51\\
    $xydx+(x+1)dy=0\\
    \frac{y}{dy}=-\frac{x+1}{xdx}\\
    -\frac{xdx}{x+1}=\frac{dy}{y}\\
    -\int\frac{xdx}{x+1}=\int\frac{dy}{y}\\
    -ln(x+1)+x+1=ln(y)+C\\$
    №52\\
    $\sqrt{y^2+1}dx=xydy\\
    \frac{dx}{x}=\frac{ydy}{\sqrt{y^2+1}}\\
    ln(x)=\sqrt{y^2+1}+C\\$
    №55\\
    $y'=\\
    \frac{dy}{dx}=3\sqrt[3]{y^2}\\
    \frac{dy}{3\sqrt[3]{y^2}}=dx\\
    \frac{1}{3}\int{\frac{dy}{y^{\frac{2}{3}}}}=\int dx\\
    \sqrt[3]{y}=x+C\\$
    №60\\
    $\frac{dz}{dx}=10^x 10^z\\
    \frac{dz}{10^z}=10^xdx\\
    \frac{10^z}{ln(10)}=-\frac{1}{ln(10)10^x}\\
    10^z=-\frac{1}{10^x}\\$
    №137\\
    $(2x+1)y'=4x+2y\\$
    решение однородного:\\
    $(2x+1)y'-2y=0\\
    \frac{1}{2}\frac{dy}{y}=\frac{1}{2}(\frac{dx}{x+\frac{1}{2}})\\
    \frac{dy}{2y}=\frac{dx}{2x+1}\\
    ln(y)=ln(x+\frac{1}{2})+ln(C)\\
    y=C(x+\frac{1}{2})\\
    (x+\frac{1}{2})(C'(x+\frac{1}{2})+C)-C(x+\frac{1}{2})=2x\\
    C'(x+\frac{1}{2})^2=2x\\
    dC=\frac{2xdx}{(x+\frac{1}{2})^2}\\
    C=2(ln(2x+1)+\frac{1}{2x+1})+C_2\\
    y=(2(ln(2x+1)+\frac{1}{2x+1})+C_2)(x+\frac{1}{2})\\$
    №139\\
    $(xy+e^x)dx-xdy=0\\
    xy-xy'=-e^x\\$
    решение однородного\\
    $xy-xy'=0\\
    dx=\frac{dy}{y}\\
    x=ln(y)-ln(C)\\
    y=Ce^x\\
    xCe^x-xC'e^x-xCe^x=-e^x\\
    xC'=1\\
    dC=\frac{dx}{x}\\
    C=ln(x)+C_2\\
    y=(ln(x)+C_2)e^x\\$
    №140\\
    $x^2y'+xy+1=0\\$
    Решение однородного\\
    $y'+\frac{y}{x}=0\\
    \frac{dy}{y}=-\frac{dx}{x}\\
    ln(y)=-ln(x)+ln(C)\\
    y=\frac{C}{x}\\
    C'x-C+C+1=0\\
    dC=-\frac{dx}{x}\\
    C=-ln(x)+C_2\\
    y=\frac{-ln(x)+C_2}{x}\\$
    \newpage
    1)уравнение вида $F(x,y,y')=0$ можно решать двумя методами\\
    \begin{tabbing}
        a)разрешшить уравнение относительно y', то есть из уравниея F(x,y,y')=0 выразить y' черех x и y
        таким образом\\
        получится одно или несколько уравний вида y'=f(x,y) и каждое из них нужно решить.\\
        б) метод введение параметра. Пусть уравение F(x,y,y')=0 можно разрешить относительно y, то есть записать
        \\ввиде y=f(x,y')\\
    \end{tabbing}
    Вводим параметр p=y' отсюда следует y=f(x,p)\\
    Прихоим к уравнению вида M(x,p)dx+N(x,p)dp=0\\
    Если решение этого ур $x=\phi(p)\\$    
    Диффиренциал-
    $x=\phi(p)\\
    y=f(\phi(p),p)\\$
    Пример:\\
    $y=x+y'-ln(y')\\
    p=y'\\
    y=x+p-ln(p)\\
    dy=dx+dp-\frac{dp}{p}\\
    pdx=dx+dp-\frac{dp}{p}\\
    dx(p-1)=dp(1-\frac{1}{p})\\
    p\not=1;\\
    dx=\frac{dp}{p}\\
    x=ln(p)+C\\
    p=Ce^x\\
    y=ln(p)+C+p-ln(p)\\
    y=C+p\\
    y=C+Ce^x\\
    y=C(e^x+1)\\
    p=1\\
    y=x+1\\$
    Дискрименантная кривая-\\
    решение $y=\phi(x)$ уравнения $F(x,y,y')=0$ называется особым если через каждую точку, кроме этого решения проходит и другое решение, имеющее в этой точке ту же касательную, что и решение $y=\phi(x)$, но не совпадющее с ним в сколько угодно малой окерсности этой точке.\\
    Если функция $F(x,y,y')$ и прозводная $\frac{\partial f}{\partial y}$ и производная $\frac{\partial f}{\partial y'}$ непрерывна то любое особое решение уравнения $F(x,y,y')=0$ удовлетворяет также уравению $\frac{\partial F(x,y,y')}{\partial y'}=0$\\

    Пример 2: найти особое решение уравнения\\
    $y=x+y'-ln(y')\\
    0=1-\frac{1}{y'}\\
    y'=1\\
    y=x+1\\
    y_1(x_0)=y_2(x_0)\\
    y_1'(x_0)=y_2'(x_0)\\
    x+1=e^{x-C}+C\\
    1=e^{x-C}\\
    x_0=C\\
    C+1=1+C\\$
    Пример 3:\\
    $(y')^2-y^2=0\\
    y'=p\\
    p^2-y^2=0\\
    pdp=ydxp\\
    p\not=0\\
    \frac{dp}{dx}=y\\
    p=\frac{dp}{dx}\\
    x=ln(p)+C\\
    p=e^{x-C}\\
    y'=e^{x-C}\\
    (e^{x-C})^2-y^2=0\\
    y=e^{\pm x-C}\\$

    












    \newpage
    Задача коши\\
    $\begin{cases}
        \frac{dy}{dx}+\rho(x)y(x)=0\\
        y(x_0)=y_0\\
    \end{cases}$\\
    доп условие, которое делает диффур корректным как задачу.\\
    Корректность\\
    1) решение существует\\
    2) решение единственно\\
    3) решение не сильно изменяется от изменния входных данных(решение непрерывно зависит от данных в некоторой разумной топологии)\\\\

    $Cx^2+2=2Cy\\
    4C+2=2C\\
    C=-\frac{1}{2}\\$\\
    $\begin{cases}
        y'+y=e^{-x}sinx\\
        y(0)=0\\
    \end{cases}$\\
    $y'+y=e^{-x}sinx\\
    \frac{dy}{dx}+y=0\\
    ln(y)+x-ln(C)=0\\
    y=Ce^{-x}\\
    C'e^{-x}-Ce^{-x}+Ce^{-x}=e^{-x}sinx\\
    C'-sinx=0\\
    \frac{dC}{dx}=sinx\\
    C_1=-cosx+C_2\\
    y=(-cosx+C_2)e^{-x}\\
    C_2=1\\$
    Ответ:$y=(1-cosx)e^{-x}$\\\\
    $\begin{cases}
        y''-y'=4x\\
        y(0)=1\\
        y'(0)=-4
    \end{cases}$\\
    $y'=p\\
    p'-p=0\\
    \frac{dp}{dy}=p\\
    ln(p)=y+C\\
    p=Ce^{y}\\
    y'=Ce^y\\
    \frac{dy}{e^y}=Cdx\\
    -e^{-y}=Cx+C_2\\
    y=-ln(-Cx-C_2)\\
    -(\frac{-C}{-Cx-C_2})'+(\frac{-C}{-Cx-C_2})=4x\\
    \frac{-C^2}{(-Cx-C_2)^2}+\frac{-C}{-Cx-C_2}=4x$
    \newpage
    №2451\\
    $y^2(y'^2+1)=1\\
    \frac{dy}{dx}=\frac{\sqrt{1-y^2}}{y}\\
    \frac{ydy}{\sqrt{1-y^2}}=dx\\
    -\sqrt{1-y^2}=x+C\\$
    №2452\\
    $y'^2-4y^3=0\\
    y'=y^\frac{3}{2}\\
    \frac{dy}{y^\frac{3}{2}}=dx\\
    -\frac{2}{\sqrt{y}}=x\\$
    №247\\
    $xy'^2=y\\
    \frac{dy}{dx}=\sqrt{\frac{y}{x}}\\
    2\sqrt{x}=2\sqrt{y}+C$
\end{document}