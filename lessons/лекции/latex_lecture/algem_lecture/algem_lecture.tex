\documentclass[a4paper]{article}
\usepackage{graphicx}
\usepackage[english,russian]{babel}
\usepackage[utf8]{inputenc}
\usepackage[T2A]{fontenc}
\usepackage{tabto}

\begin{document}
    \begin{center}
        \LARGE{Зачет по алгебере и геометрии}
    \end{center}
    \newpage
    Список вопросов 1-й семестр.

    I. Векторная алгебра и аналитическая геометрия.

    1. Линейные операции над векторами.

    2. Базис и координаты вектора.

    3. Декартовы и прямоугольные координаты точки.

    4. Уравнение линии на плоскости Оху.

    5. Уравнения окружности (неявное, параметрическое)

    6. Простейшие задачи аналитической геометрии: расстояние межу двумя точками, деление отрезка в заданном отношении, координаты середины отрезка.

    7. Определители и их свойства

    Минор и алгебраическое дополнение определителя.

    8. Коллинеарные и компланарные векторы.

    9. Ортонормированный базис и прямоугольные координаты вектора.

    10. Условие коллинеарности двух векторов в координатной форме.

    11. Скалярное произведение двух векторов: определение, свойства. Угол между двумя векторами. Условие ортогональности двух векторов.

    12. Проекция вектора на вектор.

    13. Правая и левая тройки векторов.

    14. Векторное произведение двух векторов: определение, свойства.

    15. Векторное произведение в координатной форме.

    16. Смешанное произведение векторов, определение. Смешанное произведение векторов в координатной форме.

    17. Геометрический смысл смешанного произведение векторов.

    18. Условие компланарности трех векторов.

    19. Линейные геометрические объекты: прямая на плоскости.\newline
    Виды уравнений прямой. (доказательство эквивалентности различных способов задания прямой на плоскости). Направляющий вектор прямой

    20. Угловой коэффициент и нормальный вектор прямой на плоскости Оху.

    21. Угол между прямыми на плоскости. Расстояние между прямыми на плоскости.

    22. Плоскость в пространстве. Виды уравнений плоскости и доказательство эквивалентности этих уравнений. Нормальный вектор плоскости.

    23. Расстояние между параллельными плоскостями. Расстояние от точки до плоскости. Угол между плоскостями.

    24. Уравнения прямой в пространстве. Виды уравнений прямой в пространстве. Доказательство эквивалентности различных видов уравнений прямой в пространстве.

    25. Взаимное расположение прямой и плоскости в пространстве. Угол между прямой и плоскостью.

    26. * Расстояние между скрещивающимися прямыми.

    27. * Условие принадлежности двух прямых одной плоскости.

    28. Кривые на плоскости. Уравнение кривой в декартовой системе координат.

    29. Алгебраические кривые второго порядка: эллипс, гипербола, парабола. Канонические уравнения. Основные параметры.

    30. Полярная система координат.

    31. Уравнения кривой в полярной системе координат. Построение простейших кривых.

    32. Параметрические уравнения кривой.

    33. Поверхности и кривые в пространстве.

    34. Алгебраические поверхности второго порядка. Общий вид. Канонические уравнения кривых: эллипсоид, гиперболоиды - однополостной и двуполостной, конус второго порядка, параболоиды эллиптический и гиперболический, цилиндры второго порядка - эллиптический, гиперболический и параболический. Вырожденные кривые.

    35. Классификация кривых по типу преобразования пространства.

    П. Определители и матрицы. Системы линейных уравнений.

    1. Определители. Определители второго и третьего порядка. Основные методы вычисления. Разложение определителя по любой строке и столбцу.

    2. Определители n-го порядка.

    3. Матрицы. Операции над матрицами. Обратная матрица. Матричные уравнения.

    4. Системы линейных уравнений: общий вид, матрица системы, запись в матричной форме, решение системы, совместная система, однородная система.

    5. Правило Крамера. Решение систем уравнений по правилу Крамера в матричной и покомпонентной форме.

    6. Пространство арифметических векторов. Пространство : векторы и определение линейных операций, линейная комбинация векторов, линейная (не)зависимость системы векторов.

    7. Ранг и базисный минор матрицы

    8. Теорема о базисном миноре

    9. Условие линейной независимости m векторов из пространства

    10. Условие линейной независимости n векторов из пространства

    11. Ранг матрицы. Вычисление ранга матрицы методом окаймляющих миноров.

    12. Ранг матрицы. Вычисление ранга матрицы методом элементарных преобразований.

    13. Решение произвольных систем методом Крамера.

    14. Однородные системы.

    15. Расширенная матрица системы линейных уравнений.

    16. Теорема Кронекера – Капелли.

    17. Метод последовательных исключений Жордана-Гаусса.
    \newpage
    \section{Линейные операции над векторами.}
    1)Сложение (сумма) векторов\newline
    \begin{picture}(110,50)
        \vector(0,1){10}
    \end{picture}
\end{document}