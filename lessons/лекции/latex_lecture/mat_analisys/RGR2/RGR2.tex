\documentclass[a4paper]{article}
\usepackage{graphicx}
\usepackage[english,russian]{babel}
\usepackage[utf8]{inputenc}
\usepackage[T2A]{fontenc}
\usepackage{tabto}
\usepackage{amsmath}
\usepackage{pgfplots}
\usepackage[left=25mm, top=20mm, right=10mm, bottom=10mm, nohead, nofoot]{geometry}
\begin{document}
    \begin{center} 
        \LARGE РГР\newline
        Вариант 14
    \end{center}
    
    \newpage
\section{Исследовать функцию $y=x^3+x^2$ и построить ее график.}
1)Область определения: $E(y)=(-\infty;+\infty)$ \newline
2)Асимптоты отсутствуют
\\
3)
\\
a)При y=0
\\
$x^{3}+x^{2}=0$\\
$x^{2}(x+1)=0 \Rightarrow$
\\
$\left\{
    \begin{gathered}
    x=0
    \\
    x=-1
    \end{gathered}
\right.$
\\\\
б)При x=0
\\
y=0
\\
4)Функция не является ни четной, ни ничетной
\\
5)
$y'=3x^{2}+2x$\\
$3x^{2}+2x=0$\\
$\left[
    \begin{gathered}
        x=0
        \\
        x=-\frac{2}{3}
        \end{gathered}
    \right.$\\\\
    \begin{picture}(110,20)
    \vector(1,0){100}
    \put(-80,-15){-$\frac{2}{3}$}
    \put(-40,-15){0}
    \put(-75,-2){|}
    \put(-40,-2){|}
    \put(-90,-10){+}
    \put(-55,-10){-}
    \put(-20,-10){+}
    \put(-90,5){\vector(1,1){10}}
    \put(-55,15){\vector(1,-1){10}}
    \put(-25,5){\vector(1,1){10}}
    \put(0,-5){x}
    \end{picture}\\\\\\
    $x_{min}=0$\\
    $x_{max}=-\frac{2}{3}$\\
    Монотонно возрастает $x \in(-\infty;-\frac{2}{3})\cup(0,+\infty)$\\
    Монотонно возрастаеn $x \in (-\frac{2}{3};0)$\\
    6)$y''=6x+2$\\
    $6x+2=0$\\
    $x=-\frac{1}{3}$\\
    $x \in (-\infty;-\frac{1}{3})$ Функция выпуклая\\
    $x \in (-\frac{1}{3};\infty)$  Функция вогнутая\\
    7) Асимптот нет\\
    8)График:\\
    \begin{tikzpicture}
        \begin{axis}[
            title = $x^{3}+x^{2}$,
            xlabel = {$x$},
            ylabel = {$y$},
            minor tick num = 5
        ]
        \addplot[blue]{x^2+x^3};
        \end{axis}
        \end{tikzpicture}
    \section{Исследовать функцию $y=\frac{(x-1)^{2}}{x-2}$ и построить ее график.}
    1)$E(y)=(-\infty;2)\cup(2;+\infty)$\\
    2)$x=2$ вертикальная асимптота\\
    3)
    a) при $y=0$ $x=1$\\
    б) при $x=0$ $y=-1/2$\\
    4)функция является ни четной, ни ничетной\\
    5)$y'=\frac{2(x-1)(x-2)-(x-1)^{2}}{(x-2)^2}=
    \frac{2x^2-6x+4-x^2+2x-1}{(x-2)^2}=
    \frac{x^2-4x+3}{(x-2)^2}=
    \frac{(x-1)(x-3)}{(x-2)^2}$\\
    прировняем производную к нулю\\
    $\frac{(x-1)(x-3)}{(x-2)^2}=0$\\
    \begin{picture}(110,20)
        \vector(1,0){130}
        \put(-40,-15){3}
        \put(-75,-15){2}
        \put(-110,-15){1}
        \put(-40,-2){|}
        \put(-75,-2){|}
        \put(-110,-2){|}
        \put(-125,-10){+}
        \put(-90,-10){-}
        \put(-55,-10){-}
        \put(-20,-10){+}
        \put(-125,5){\vector(1,1){10}}
        \put(-90,15){\vector(1,-1){10}}
        \put(-55,15){\vector(1,-1){10}}
        \put(-25,5){\vector(1,1){10}}
        \put(0,-5){x}
        \end{picture}\\\\\\
        $x_{max}=1$\\
        $x_{min}=3$\\
        Функция возрастает при $x \in (-\infty;1)\cup(3;+\infty)$\\
        Функция убывает при $x \in(1;2)\cup(2;3)$\\\\
        6)$y''=\frac{((x-3)+(x-1))(x-2)^2+2(x-1)(x-3)(x-2)}{(x-2)^4}
        =\frac{(2x-4)(x-2)+2(x-1)(x-3)}{(x-2)^3}
        =\frac{2x^2-8x+8-2x^2+8x-10}{(x-2)^3}
        =\frac{-2}{(x-2)^3}$\\
        $\frac{-2}{(x-2)^3}=0$\\
        При $x \in (-\infty;2)$ вогнута\\
        При $x \in (2;+\infty)$ выпукла\\
        7)вертикальные и горизонтальные ассимптоты\\
        $\lim\limits_{x \to \infty} (kx+b+f(x))$\\
        $k=\lim\limits_{x \to \infty} \frac{f(x)}{x}$\\
        $k=\lim\limits_{x \to \infty} \frac{\frac{(x-1)^{2}}{x-2}}{x}
        =\lim\limits_{x \to \infty} \frac{(x-1)^{2}}{x(x-2)}=1$\\
        $b=\lim\limits_{x \to \infty} f(x)-kx$\\
        $b=\lim\limits_{x \to \infty} \frac{(x-1)^{2}}{x-2}-x
        =\lim\limits_{x \to \infty} \frac{1}{x-2}=0$\\
        Получается уравнение y=x наклонной асимптоты\\
        Вертикальная асимптота x=2\\
        \begin{tikzpicture}
            \begin{axis}[
                title = $\frac{(x-1)^2}{x-2}$,
                xlabel = {$x$},
                ylabel = {$y$},
                minor tick num = 5
            ]
            \addplot[red]{x};
            \addplot[blue]{((x-1)^2)/(x-2)};
            \end{axis}
        \end{tikzpicture}
        \section{Исследовать функцию $y=\frac{1}{(x-1)e^x}$ и построить ее график.}
        1)$E(y)=(-\infty;1)\cup(1;\infty)$\\
        2)функция общего вида\\
        3)$x=0, y=-1$\\
        $y=0$ пересечений нет\\
        4)
        $y'=\frac{e^{-x}(x-1)-e^{-x}}{(x-1)^2}
        =\frac{-e^{-x}x}{(x-1)^2}$\\
        $\frac{-e^{-x}x}{(x-1)^2}=0$\\
        Убывает при $x \in(0,1)\cup(1,=\infty)$\\
        Возрастает при $x \in(-\infty,0)$\\
        5)$y''=\frac{(e^{-x}x-e^{-x})(x-1)^2+2e^{-x}x(x-1)}{(x-1)^4}
        =\frac{ e^{-x}((x-1)^2+2x)}{(x-1)^3}=
        \frac{x^2+1}{(x-1)^3e^x}$\\
        $\frac{x^2+1}{(x-1)^3e^x}=0$\\
        Вогнута при $x \in (1;+\infty)$\\
        Выпукла при $x \in (-\infty;1)$\\
        6)
        $\lim\limits_{x \to \infty} (kx+b+f(x))$\\
        $k=\lim\limits_{x \to \infty} \frac{f(x)}{x}$\\
        $k=\lim\limits_{x \to \infty} \frac{1}{(x-1)e^xx}=0$\\
        $b=\lim\limits_{x \to \infty} f(x)-kx$\\
        $b=\lim\limits_{x \to \infty} \frac{1}{(x-1)e^x}-0x
        =\lim\limits_{x \to \infty} \frac{1}{(x-1)e^x}=0$\\
        $y=0$ горизонтальная асимптота\\
        $x=1$ вертикальная асимптота\\
        7)график\\
        \begin{tikzpicture}
            \begin{axis}[
                title = $\frac{1}{(x-1)e^x}$,
                xlabel = {$x$},
                ylabel = {$y$},
                minor tick num = 5
            ]
            \addplot[red]{0};
            
            \addplot coordinates{
                (1,20) (1,-20)};
            \addplot[blue]{(1/((x-1)*e^x)};
            \end{axis}
        \end{tikzpicture}
        \section{Найти наименьшее значнении функции $y=\frac{3}{x+1}-\frac{3}{x-3}+2$ на отрезке [0,2].}
        $y'=\frac{-3}{(x+1)^2}+\frac{3}{(x-3)^2}
        =\frac{-3x^2+18x-27+3x^2+6x+3}{(x+1)^2(x-3)^2}
        =\frac{24x-24}{(x+1)^2(x-3)^2}$\\
        $\frac{24x-24}{(x+1)^2(x-3)^2}=0$\\
        $\frac{x-1}{(x+1)^2(x-3)^2}=0$\\
        \begin{picture}(110,20)
            \vector(1,0){130}
            \put(-40,-15){3}
            \put(-75,-15){1}
            \put(-110,-15){-1}
            \put(-40,-2){|}
            \put(-75,-2){|}
            \put(-110,-2){|}
            \put(-125,-10){-}
            \put(-90,-10){-}
            \put(-55,-10){+}
            \put(-20,-10){+}
            \put(-125,15){\vector(1,-1){10}}
            \put(-90,15){\vector(1,-1){10}}
            \put(-55,5){\vector(1,1){10}}
            \put(-25,5){\vector(1,1){10}}
            \put(0,-5){x}
        \end{picture}\\\\\\
        $x_{min}=1$
\end{document}
